\nonumber		%% für align

\newcommand{\WikiURL}{http://de.wikipedia.org/wiki}
\newcommand{\tableprintACSandACL}[1]{\acs*{#1} & \acl*{#1}}
\renewcommand{\printSectionListOfAbbreviationsChemistry}{
\section*{Abkürzungsverzeichnis\protect\footnote{Verzeichnis aller im Text, ausgenommen Tabellen,
benutzten chemischen
Elemente, sortiert nach ihrer Ordnungszahl und Moleküle. Die Elemente und Moleküle wurden beim
ersten Auftreten ausgeschrieben und die Abkürzungen in Klammern dahinter gesetzt. Alle folgenden
Male wurde nur die Abkürzung benutzt. In Überschriften wurde davon unabhängig die Langform
benutzt.}}
%% Section ist größer als Inhaltsverzeichnis
\addcontentsline{toc}{section}{Abkürzungsverzeichnis}
}

\newcommand{\donkeybridge}[1]{%
{\centering%

Eselsbrücke #1

}}

\usepackage{chemarrow}

\newcommand{\InMathisHeft}[1]{\ifDraft{%
\texttt{Begin Mathis Heft \\ Das folgende ist aus dem Heft von Mathis (frühere Klasse)... \\ #1}

\texttt{End Mathis Heft}
}{}}

\ifDraft{
%	\cfoot{Version: \printversion}
	\usepackage[draft,nomargin,inline]{fixme}
	\fxusetheme{colorsig}\renewcommand{\germanlistfixmename}{Verzeichnis der Anmerkungen}
	}{
	\usepackage{fixme}
}

\newcounter{cVersuch}
%\newcommand{\Expname}{\textcolor{bleu}{Experiment \arabic{cVersuch}:}}
\newcommand{\Expname}{Experiment \arabic{cVersuch}:}
\newcommand{\cVersuch}[2]{
  \stepcounter{cVersuch}
  \sixIF{#1}{\section{\Expname ~#2}}
    {\subsection{\Expname ~#2}}
    {\subsubsection{\Expname ~#2}}
    {\pdfbookmark[1]{Experiment \arabic{cVersuch}: #2}{Experiment_\arabic{cVersuch}}
    \section*{\Expname ~#2}}
    {\pdfbookmark[2]{Experiment \arabic{cVersuch}: #2}{Experiment_\arabic{cVersuch}}
    \subsection*{\Expname ~#2}}
    {\pdfbookmark[3]{Experiment \arabic{cVersuch}: #2}{Experiment_\arabic{cVersuch}}
    \subsubsection*{\Expname ~#2}}
    {n.d.}
}

%% chemfig
\definesubmol{CH2-Ring6}{C(-[::-30]H)(-[::-90]H)}
\definesubmol{COH}{C(-[::90]H)(-[::-90]OH)}
