\section{Aldehyde}
\begin{description}
	\item[Untergruppe:] Alkanale \\
		Alkanale sind kettenförmige organische Verbindungen mit einer
		Aldehydgruppe im Molekül. Sie sind Derivate der Alkane.
	\item[Einige Vertreter:]
	\begin{description}
		\item[\ac{CH2O}:] \chemfig{H-C(=[::45]\lewis{02,O})(-[::-45]H)} \hfil beziehungsweise \ce{HCHO}
		\item[\ac{C2H4O}:] \chemfig{H-!{CH2}-C(=[::45]\lewis{02,O})(-[::-45]H)}
			\hfil beziehungsweise \ce{CH3-CHO}
		\item[\ac{C3H6O}:] \chemfig{H-!{CH2}-!{CH2}-C(=[::45]\lewis{02,O})(-[::-45]H)}
			\hfil beziehungsweise \ce{CH3-CH2-CHO}
	\end{description}
\end{description}

\subsection{Nachweise der Aldehydgruppen}
\cVersuch{3}{Nachweis für Aldehyde}
\begin{description}
	\item[Aufbau:] Ein Becherglas, gefüllt mit konzentriertem \ac{C2H5OH}
		und fuchsinschwefliger Säure (Schiffsches Reagenz).
		Einem Bunsenbrenner und ein \ac{29}-Streifen.
	\item[Durchführung:]~
	\begin{enumerate}
		\item \ac{29}-Streifen in der Flamme oxidieren lassen.
		\item In die Flüssigkeit halten.
	\end{enumerate}
	\item[Beobachtung:] Die Flüssigkeit färbt, wenn das \ac{CuO} hinzukommt.
		Es geht von farblos zu Rot-Lila.
		Das \ac{CuO} wird wieder zu \ac{29} (es wird wieder glänzend).
	\item[Gleichung:] \chemfig{H-!{CH2}-C(-[::90]H)(-[::-90]\textcolor{red}{H})
			-\lewis{26,O}-\textcolor{red}{H}} \chemsign{+}
		Cu\textcolor{blue}{O} \chemsign{\ce{->}} \ce{Cu} \chemsign{+}
		\textcolor{red}{\ce{H2}}\textcolor{blue}{O} \chemsign{+}
		\chemfig{H-!{CH2}-C(=[::45]\Lewis{02,O})(-[::-45]H)} \\[0.5em]
		Es entsteht \ac{29}, \ac{H2O} und \ac{C2H4O}. %% Alkanal
\end{description}

\subsubsection{Gleich zu Experiment \arabic{cVersuch} nur mit 2-Propanol}
\chemfig{H-!{CH2}-C(-[::90,2]\lewis{24,O}(-H))(-[::-90]H)-!{CH2}-H}
\chemsign{+} \ce{CuO} \chemsign{\ce{->}}
\ce{Cu} \chemsign{+} \ce{H2O} \chemsign{+}
\chemfig{H-!{CH2}-C(=[::90]\lewis{13,O})-!{CH2}-H}

\cVersuch{3}{Fehling-Probe}
\begin{description}
	\item[Aufbau:] Fehling 1 (\ac{CuSO4}, ist hellblau)
		und Fehling 2 (\ac{C4H4KNaO6}, ist farblos) in ein Becherglas geben.
		Die zwei Flüssigkeiten werden zusammen Dunkelblau (Königsblau).
	\item[Durchführung:] Die zu Testende Substanz dazugeben, hier \ac{C6H12O6}.
	\item[Beobachtung:] Die Flüssigkeit färbt sich über Braun zu Orange.
\end{description}

\cVersuch{3}{Tollensprobe (Silberspiegel-Probe)}
\begin{description}
	\item[Aufbau:] \ac{AgNO3}, \ac{NH4OH}, \ac{NaOH}, \ac{C6H12O6}, Dreifuß, Bunsenbrenner,
		Becherglas, Reagenzglas.
	\item[Durchführung:] \ac{C6H12O6} in ein Reagenzglas geben,
		dazu \ac{AgNO3}, \ac{NH4OH} und \ac{NaOH}. Nun langsam erhitzen.
		Die Flüssigkeit darf nicht sieden.
		Außerdem sollte die Flüssigkeit nicht unnötig geschüttelt werden.
	\item[Beobachtung:] Die Flüssigkeit wird erst braun,
		dann setzt sich innen am Reagenzglas \ac{47} ab.
\end{description}

\fxwarning{Hier fehlt Exp. \acl{C2H4O} und \ac{CuO} reagiert zu \acl{29} und \acl*{C2H4O2}}

\renewcommand{\longtableheader}{\multicolumn{1}{l}{\textbf{Name}}
& \multicolumn{1}{l}{\textbf{Endungen}}
\\
}
\begin{longtable}{ll}
	\longtableheader
	\endfirsthead
	\longtableheader
	\endhead
	\caption{Übersicht über die behandelten funktionellen Gruppen}
	\endlastfoot
	\multicolumn{2}{r}{\longtableendfoot} \\
	\endfoot

	\hyperref[sec:Alkane:Definition]{Alkane}		& -an \\
	\hyperref[sec:Alkene]{Alkene}					& -en \\
	\hyperref[sec:Alkine]{Alkine}					& -in \\
	\hyperref[sec:Alkanole]{Alkanole}				& -ol \\
	\hyperref[sec:Alkanole]{Alkohole}				& -ol\fxwarning{true ???} \\
	\href{\WikiURL/Ketone}{Ketone}					& -on \\
	\href{\WikiURL/Ether}{Ether}					& -ether \\
	\href{\WikiURL/Alkoholate}{Alkoholate}			& \\
	\href{\WikiURL/Alkanole}{Alkanole}				& \\
\end{longtable}
