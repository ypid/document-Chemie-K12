\section{Alkene (Olefine)}
\label{sec:Alkene}
Alkene sind ungesättigte azyklische oder aliphatische (Kettenförmige) Kohlenwasserstoffe
mit einer Doppelbindung zwischen zwei \ac{6}-Atomen. Alle anderen Bindungen sind
Einfachbindungen.

\ac{C2H4} ist das erste mögliche Alken.

\ac{C2H4}: \ce{CH2=CH2} beziehungsweise \chemfig{C(-[::135]H)(-[::-135]H)=C(-[::45]H)(-[::-45]H)}

\ac{C3H6}: \ce{CH2=CH-CH3}

Die Allgemeine Summeformel für die Alkene ist: $\mathrm{C_nH_{2n}}$

\begin{itemize}
	\item Alkane gehen in den Raum (drei Dimensionen).
	\item Alkene breiten sich in der Fläche aus (zwei Dimensionen).
	\item Alkine sind gerade Gebilde (eine Dimensionen).
\end{itemize}

\subsection{Doppelbindungsisometrie}
\begin{description}
	\item[n-\acl{C4H8}:] \acused{C4H8} 1-\acl{C4H8}\hspace{6ex}
		$\left.
		\begin{array}{cc}	%% für mehrzeiligen Text nötig
			\ce{CH2=CH-CH2-CH3} \\
			\ce{CH3-CH2-CH=CH2}
		\end{array}
		\right\}
		\text{gleicher Stoff}$
	\item[Das Isomere:] 2-\acl{C4H8}\hspace{2.63ex} \ce{CH3-CH=CH-CH3}
\end{description}

\ifDraft{
\begin{itemize}
	\item Die Siedetemperatur ist extrem niedrig. Sie nimmt bei zunehmender Kettenlänge zu.
		\fxwarning{true ???}
\end{itemize}
}{}

\cVersuch{2}{Alken und Halogen}
\begin{description}
	\item[Aufbau:] 1-Hexen, \ac{Br2}, Becherglas,
		angefeuchtetes Indikatorpapier (Lackmuspapier) ist oben am Becherglas befestigt,
		Pipette, Dreifuß, Bunsenbrenner
	\item[Beobachtung:]~
	\begin{itemize}
		\item Sofortige Reaktion.
		\item Das rote \ac{Br2} entfärbt sich schlagartig.
		\item Das Indikatorpapier färbt sich nicht.
	\end{itemize}
	\item[Erklärung:] Es kommt zu einer homolytischen Spaltung.
		Das \ac{Br2} lagert sich an der Bindung der \ac{6}-Atome an.
	\item[Formel:] \ce{CH2=CH-(CH2)3-CH3 +} \lewis{246,Br}\ce{-}\lewis{026,Br}
		\ce{-> CH2Br-CHBr-(CH2)3-CH3}
\end{description}

\subsection{Addition}
Die Addition ist eine chemische Reaktion, bei der sich an eine Mehrfachbindung, Atome oder
Atomgruppen, unter Aufspaltung einer Bindung, anlagern.

\subsection{Alkene bilden eine homologe Reihe}
\Siehe{sec:homologe_Reihe}.

\vspace{-5ex}
\begin{align}
\ac{C2H4} + \ac{Cl2} & \ce{->} \text{1,2-Dichlorethan} \\
\ce{CH2=CH2 +} \lewis{246,Cl}\ce{-}\lewis{026,Cl} & \ce{-> CH2Cl-CH2Cl} \\
\ac{C3H6} + \ac{H2} & \ce{->} \text{\ac{C3H8}}\\
\ce{CH2=CH-CH3 + H-H}		& \ce{-> CH3-CH2-CH3}
\end{align}

Durch Addition von \ac{H2} kann man aus Alkenen Alkane herstellen.
Diese Reaktion heißt auch Hydrierung.

Hydrierung ist eine Anlagerung von \ac{H2} an Mehrfachbindungen

2-Penten + \ac{HCl}

\begin{minipage}{0.45\hsize}
	\chemfig{H-!{CH2}-C(-[::-90]H)=C(-[::90]H)-!{CH2}-!{CH2}-H}\chemsign{+}%
	\ce{H-\lewis{026,Cl}}
%	\ce{H-Cl}
\end{minipage}\hfill
\begin{minipage}{0.45\hsize}
	\ce{->} \chemfig{H-!{CH2}-C(-[::90]\lewis{024,Cl})(-[::-90]H)-!{CH2}-!{CH2}-!{CH2}-H} \\
	oder \\
	\ce{->} \chemfig{H-!{CH2}-!{CH2}-C(-[::90]H)(-[::-90]\lewis{024,Cl})-!{CH2}-!{CH2}-H}
\end{minipage}

Bei jeder Addition entsteht nur ein Reaktionprodukt,
aber je nach Zusammensetzung der Ausgangsprodukte, gibt es verschieden Möglichkeiten
der Endprodukte.

\begin{description}
	\item[Alkane:] $\mathrm{C_nH_{2n+2}}$, gesättigte, Einfachbindungen,
		Reaktionstyp: Substitution
	\item[Alkene:] $\mathrm{C_nH_{2n}}$, ungesättigt, eine Doppelbindung
		alle anderen sind Einfachbindungen, Reaktionstyp: Addition
	\item[Alkine:] $\mathrm{C_nH_{2n-2}}$, ungesättigt, eine Dreifachbindung
		alle anderen sind Einfachbindungen, Addition
\end{description}

Alkene mit mehr als einer Doppelbindung heißen: Diene, Triene~\dots

Die Doppelbindungen können in unterschiedlichen Varianten angeordnet sein.

\begin{description}
	\item[kumulierte Bindung:]
		\chemfig{C(-[::135]H)(-[::-135]H)=C=C=C=C(-[::45]H)(-[::-45]H)} \\[0.5ex]
		Viele Doppelbindungen nebeneinander (instabil).
	\item[konjugierte Bindung:]
		\chemfig{C(-[::135]H)(-[::-135]H)=C(-[::-90]H)
		-C(-[::90]H)=C(-[::-90]H)-C(-[::45]H)(-[::-45]H)} \\[0.5ex]
		Zwischen den Doppelbindungen ist jeweils eine Einfachbindungen
		(etwas stabiler).
	\item[isolierte Bindung:]
		\chemfig{C(-[::135]H)(-[::-135]H)=C(-[::-90]H)
		-!{CH2}-C(-[::90]H)=C(-[::45]H)(-[::-45]H)} \\[0.5ex]
		Große Entfernung zwischen den Doppelbindungen (am stabilsten).
\end{description}

Je weniger Doppelbindungen desto stabiler.
