\section{Einführung}

\begin{description}
	\item[Jöns Jacob Berzelius:] Lehrmeinung bis ins 19. Jahrhundert.
	\begin{itemize}
		\item Organische Chemie \\
			Organische Stoffe werden nur in der belebten Natur gebildet,
			mit Hilfe einer göttlichen Lebenskraft (lateinisch vis vitalis)
		\item Anorganische Chemie \\
			Anorganische Stoffe sind Stoffe der unbelebten Natur: Salze, Erze, Gesteine
	\end{itemize}
	\item[Friedrich Wöhler:] stellte 1829 \ac{CH4N2O} synthetisch aus \ac{CHNO} und \ac{NH3} her.
		Damit ist die Meinung von Jöns Jacob Berzelius widerlegt.
		\begin{center}
			\chemfig{[6]\lewis{13,O}=C(-[::45]NH_2)(-[::-45]NH_2)} %\hspace{1,5cm}\ac{CH4N2O}	%% ,,,1
		\end{center}
	\item[Neue, moderne Lehrmeinung:] organische Chemie ist die Chemie der Kohlenstoffverbindungen
		außer \ac{H2CO3}, \ac{CO}, \ac{CO2} und Karbonate (Salze der \ac{H2CO3}). \\
		Alle organischen Stoffe verbrennen zu \ac{6}, \ac{CO2} oder \ac{CO}, \ac{H2}.
%	\item[Eigenschaften von \acl{6}:]
%		\begin{itemize}
%			\item Element der \RM{4} Hauptgruppe (4 Außenelektronen)
%			\item reaktionsfreudig
%			\item kann Bindungen mit sich selbst eingehen
%		\end{itemize}
\end{description}

Die Vielfalt der organischen Stoffe resultiert aus den verschiedenen Verbindungsformen.
\begin{multicols}{2}
\begin{enumerate}
	\item Lewis-Schreibweise von \ac{6}:~ \lewis{0.2.4.6.,C}
	\item Kette \\[0.8ex]
		\chemfig{H-!{CH2}-!{CH2}-!{CH2}-H}
	\item Verzweigung \\[0.8ex]
		\chemfig{H-!{CH2}-C(-[::+90]H)(-[::-90,2]C(-H)(-[::-90]H)-[::0]H)-!{CH2}-H}
	\item Ringe \\[0.8ex]
		\chemfig{[:30]C(-[::210]H)(-[::150]H)*6(-!{CH2-Ring6}
			-!{CH2-Ring6}-!{CH2-Ring6}-!{CH2-Ring6}-!{CH2-Ring6}-)}
\end{enumerate}
\begin{itemize}
	\item Es bilden sich immer wieder neue Stoffe mit anderen Eigenschaften
	\item Es entstand nach der Entdeckung von Wöhler ein ganz neuer Industriezweig:
		Zeitalter der Kunststoffe
\end{itemize}
\end{multicols}

\begin{itemize}
	\item An die \acl{6}atome kann sich \ac{H2} binden. Es entstehen die
		Kohlenwasserstoffe.
\end{itemize}


\InMathisHeft{
\begin{multicols}{2}
\begin{enumerate}
	\item Kette \\
		\chemfig{\lewis{2.4.6.,C}-\lewis{2.6.,C}-\lewis{2.6.,C}-\lewis{2.6.,C}-\lewis{0.2.6.,C}}
	\item Verzweigung \\[0.8ex]
		\chemfig{\lewis{4.6.,C}-[2]\lewis{2.4.,C}-\lewis{2.6.,C}-\lewis{0.2.,C}-[6]\lewis{0.4.6.,C}}
	\item Ringe \\[0.8ex]
		\chemfig{\lewis{4.6.,C}*6(-\lewis{5.7.,C}-\lewis{0.6.,C}-\lewis{0.2.,C}-\lewis{1.3.,C}-\lewis{2.4.,C}-)}
\end{enumerate}
\end{multicols}
}
