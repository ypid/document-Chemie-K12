\section{Alkine (Acetylene)}
\label{sec:Alkine}
Alkine sind ungesättigte, organische Verbindungen, azyklisch/aliphatisch,
mit einer Dreifachbindung.
Alle anderen Verbindungen sind Einfachbindungen.

Das kleinst Mögliche Molekül ist Ethen.

\begin{description}
	\item[\ac{C2H2}:] \ce{H-C#C-H}
	\item[\ac{C3H3}:] \chemfig{H-C~C-!{CH2}-H}
\end{description}

Auch die Alkine bilden eine homologe Reihe.
\fxwarning{Hier fehlt vermutlich etwas}

\section{Einige zyklische Kohlenwasserstoffe}
Friedrich August Kekulé von Stradonitz, ein deutscher Chemiker, setzte sich mit dem Problem auseinander,
dass einige Stoffe mit gleicher Summenformel andere Reaktionen hervorbringen.

\renewcommand{\longtableheader}{\multicolumn{1}{c}{\textbf{\ce{C6H12}}} &
& \multicolumn{1}{c}{\textbf{\ce{C6H10}}} &
& \multicolumn{1}{c}{\textbf{\ce{C6H6}}}
\\[0.8ex]}
\begin{longtable}{p{0.27\hsize}cp{0.27\hsize}cp{0.27\hsize}}
	\longtableheader
	\endfirsthead
	\longtableheader
	\endhead
	\caption{Einige zyklische Kohlenwasserstoffe}
	\endlastfoot
	\multicolumn{5}{r}{\longtableendfoot} \\
	\endfoot

	\chemfig{[:30]C(-[::210]H)(-[::150]H)*6(-!{CH2-Ring6}
			-!{CH2-Ring6}-!{CH2-Ring6}-!{CH2-Ring6}-!{CH2-Ring6}-)}
	& & \chemfig{[:30]C(-[::210]H)(-[::150]H)*6(-!{CH2-Ring6}
			-C(-[::-30]H)=C(-[::-90]H)-!{CH2-Ring6}-!{CH2-Ring6}-)}
	& & \chemfig{[:30]C(-[::180]H)*6(-C(-H)
			=C(-H)-C(-H)=C(-H)-C(-H)=)} \\[8.5ex]
	Cyclohexan		& & Cyclohexen		& & 1,3,5-Cyclohexantrien \\
	Es addiert \ac{Br2} nicht (nicht erwartet)	& & \ac{Br2} wird addiert (erwartet)
	& & Es addiert \ac{Br2} nicht (nicht erwartet) \\
	& & \multicolumn{3}{c}{\emph{Mesomere Grenzstruktur}} \\
	& & \multicolumn{3}{c}{(theoretisch möglich und völlig richtig,} \\
	& & \multicolumn{3}{c}{aber in der Realität nicht vorhanden.)} \\
\end{longtable}
%Benzdring (irreführender Name)

\begin{description}
	\item[\ac{C6H6}:] \chemfig{[:30]C(-[::180]H)**6(-C(-H)-C(-H)-C(-H)-C(-H)-C(-H)-)}
		\hspace{3em}beziehungsweise\hspace{3em} \chemfig{**6(------)} \\[0.8ex]
		Das nicht addieren von Benzol mit \ac{Br2} wird so erklärt,
		dass je ein Außenelektron der sechs \ac{6}-Atome sich in einem Kreis bewegt
		und diese Außenelektronen somit allen sechs \ac{6}-Atomen gehören.
		Diesen Ring nennt man $\pi$-Elektronensextett. \\
		Wegen des angenehmen Duftes heißen sie Aromaten.
\end{description}

\subsection{Reaktion von Cyclohexan}
\begin{minipage}{10.7em}
\chemfig{[:30]C(-[::210]H)(-[::150]H)*6(-!{CH2-Ring6}
	-!{CH2-Ring6}-!{CH2-Ring6}-!{CH2-Ring6}-!{CH2-Ring6}-)}
\end{minipage}
\begin{minipage}{11.8em}
\ce{+\,} \ce{\lewis{246,Br}-\lewis{026,Br}}
\ce{->} \ce{H-\lewis{026,Br}} \ce{\,+}
\end{minipage}
\begin{minipage}{10em}
\chemfig{[:30]C(-[::210]H)(-[::150]H)*6(-!{CH2-Ring6}
	-!{CH2-Ring6}-C(-[::-30]H)(-[::-90]\lewis{026,Br})-!{CH2-Ring6}-!{CH2-Ring6}-)}
\end{minipage}

Es entsteht \ac{HBr} und 1-Monobromcyclohexan.

\subsection{Reaktion von Cyclohexen}
\begin{minipage}{10.7em}
\chemfig{[:30]C(-[::210]H)(-[::150]H)*6(-!{CH2-Ring6}
	-C(-[::-30]H)=C(-[::-90]H)-!{CH2-Ring6}-!{CH2-Ring6}-)}
\end{minipage}
\begin{minipage}{7em}
\ce{+\,} \ce{\lewis{246,Cl}-\lewis{026,Cl}}
\ce{->}
\end{minipage}
\begin{minipage}{10em}
\chemfig{[:30]C(-[::210]H)(-[::150]H)*6(-!{CH2-Ring6}
	-C(-[::-30]H)(-[::-90]\lewis{046,Cl})-C(-[::-30]\lewis{024,Cl})(-[::-90]H)
	-!{CH2-Ring6}-!{CH2-Ring6}-)}
\end{minipage}

Es entsteht 1,2-Dichlorcyclohexan.
